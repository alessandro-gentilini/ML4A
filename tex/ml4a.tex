\documentclass[12pt, letterpaper]{article}
\usepackage{xcolor}

\newcommand{\documentname}{\textsl{Note}}
\newcommand{\todo}[1]{\textcolor{red}{#1}}  % gotta have \usepackage{xcolor} in main doc or this won't work

\begin{document}

\section*{Machine learning for astronomers}

\paragraph{Abstract:} foo and bar

\section{Who are you?}

I am assuming that anyone reading this \documentname\ has a
significant fraction of the following properties:

\section{Do we need machine learning in astronomy?}

\section{Basic taxonomy of machine-learning tasks}

...Warning: This \documentname\ will be extremely non-linear, because
the connections across different subjects are myriad. If you know what
you are doing, and all you are looking for is my polemical advice, you
want to skip to \todo{where?}. If you just want to skim the relevant
algorithm names to get some ideas for some application you have in
mind, you can just read (or skim) \todo{what?}.

...In the next sections we are going to give canonical algos (meaning:
algos everyone should know, because they are gorgeous and simple) and
some pointers to more sophisticated stuff. Then we will talk about
statistical, conceptual, and mathematical considerations that cut
across applications and algorithms.

\section{Classification}

...Supervised. In astro: Star--galaxy, high-redshift quasars, cosmic rays.

...SVM

...kSVM. Concept of non-parametrics, to be discussed below.

...Rando Forest

...NNs, CNNs, RNNs, GANs, etc

...What is classification to a bayesian, and how does this relate?

\section{Regression}

...Supervised; relationship to classification. In astro: The Cannon. Kepler and LIGO systematics.

...Linear regression!

...Rando Forest, CNNs, and all the stuff from last section.

...Gaussian Processes (and we will elaborate more later). Likelihood function appears!

\section{Dimensionality reduction}

...Unsupervised. In astro: spectral templates. Feature engineering.

...PCA!

...kPCA. Why don't astronomers use this? The Dual problem problem.

...PPCA, FA, and latent-variable models like GPLVM

...Manifold-learning and t-SNE. Relationships to visualization and feature engineering (more below).

...auto-encoders: Probably useful!

...GANs

\section{Clustering}

...Unsupervised. In astro: SNe types, spectral types.

...k-means!

...[Here I am weakest]

...Using classifiers or regressions to find outliers. (The opposite of clustering.)

\section{Density estimation}

...Unsupervised. In astro: Comparing populations. Pseudo-LFs and pseudo-posteriors. Other things?

...GMM with E-M algo!

...Relationship to clustering and various confusions around that.

...XD, and why we care. And how did this become a classification algorithm? Connections there.

...RNADE

...GANs: In what sense do these estimate density?

\section{Rectangular data}

...format

...but also issues with missing data (to return below)

...kinds of data that can't be made rectangular (trivially). Specifically mention variable-star light curves.

...importance of feature engineering if you must go rectangular.

...importance of distance metrics if you can't!

\section{Distance metrics}

...Most ML methods are nearest-neighbor methods, in some sense!

\section{Statistical principles}

...train, validate, and test.

...concepts of statistical conservatism

...stationarity requirements

\section{Probabilistic methods}

\section{Concept of non-parametrics}

\section{The kernel trick}

\section{Gaussian processes}

\section{Active learning}

\section{Feature engineering}

\section{Interpretability}

\section{Generalizability}

\section{Unsolicited advice}

...only GPs are fully probabilistic and non-parametric

...model nuisances

...model or locate outliers

...operational contexts

...emulators

\paragraph{Acks}
It is a pleasure to thank
  Ross Fadely (Insight),
  Rob Fergus (NYU),
  Dan Foreman-Mackey (Flatiron),
  Jennifer Hill (NYU),
  Yann LeCun (NYU),
  Sam Roweis (deceased),
  and
  Bernhard Sch\"olkopf (MPI-IS)
for teaching me everything I know in this arena.

\end{document}
